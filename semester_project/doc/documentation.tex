\documentclass{article}

\usepackage[english]{babel}
\usepackage[utf8]{inputenc}
\usepackage{amsfonts}
\usepackage{listings}
\usepackage{hyperref}
\usepackage{xcolor}

\lstset{language=c++}

\hypersetup{
    colorlinks=true,
    linkcolor=blue,
    filecolor=magenta,
    urlcolor=cyan,
    }


\author{Adrián Habušta}
\title{Game Boy emulator - Programming Documentation}
\date{\today}

\begin{document}
    \maketitle
    \newpage

    \tableofcontents
    \newpage

    \section{Specification}
        \subsection{Summary}
            The project is meant to be a mostly-accurate emulator for the
            original Game Boy console produced by Nintendo.

        \subsection{Detailed}
            Game Boy games were distributed as a cartridge with, at the very
            least, some ROM that contains the code, graphics and audio of the
            game. The CPU works with this data and passes it to other components
            to produce output for the user. Our program needs to accept such a
            ROM as a file, and run it as the Game Boy would. Some games used
            more complicated memory patterns, such as having bigger ROMs and
            using banking to map them onto the limited size the CPU could
            access, and even sometimes having extra RAM. This RAM was sometimes
            backed with a battery to allow storing game saves.

    \section{Components}
        The emulator consists of classes representing each major component and
        an emulator class to tie them all together.
        \subsection{CPU}
            The processor is emulated by creating an instance of the \\
            \verb|central_processing_unit| class, which contains
            registers and a jump table that executes instructions. Every
            instruction is cycle accurate, but not precisely hardware accurate.
            For example, some instructions that use 16-bit registers write to
            one half of the register during one cycle, and to the other during
            another cycle. This is not emulated, since nothing except the
            processor can see these internal registers anyway. The cycles in
            which the memory is accessed are machine cycle accurate, but they
            might not be t-cycle accurate (more info on machine cycles vs
            t-cycles) can be found in the \hyperref[sec:furtherReading]{Further Reading}
            section.

            The CPU uses a helper class to send information to the emulator that a
            STOP instruction has been encountered.

            Interrupts to the CPU are sent by different components using
            callbacks that are passed to said components by the emulator at
            creation.

        \subsection{Memory mapper}
            The memory is separated into regions based on the Game Boy memory map.
            Writing/reading from these regions by the CPU initiates an m-cycle,
            which starts computing one m-cycle for each other component, before
            returning control to the CPU. This approach allows for fairly accurate
            emulation of component timings. The mapper doesn't have it's own
            memory, it only calls getters/setters for specific regions/registers.

        \subsection{Cartridge memory controller}
            This component is unfinished. It contains the basic logic to support
            different types of Cartridge MBCs (Memory Bank Controllers) but only
            the simplest type (No MBC) is implemented as of yet. This component
            allows reading from game ROM, and later will also allow sending commands
            to the MBC (through writing to the ROM), and interacting with any
            potential RAM in the emulated cartridge.

        \subsection{PPU}
            An acronym for the Pixel Processing Unit. This component is
            something like a GPU. It is not emulated 100\% correctly, because
            the timing logic of the PPU is very complex, and few games need to
            have it emulated precisely. Every t-cycle the ppu draws one pixel
            to a framebuffer using a fairly complex method to decide what to
            draw. This framebuffer is then rendered to the screen during the
            VBlank period. The different periods of the PPU are emulated using
            a simple state machine.


    \section{Rendering}
        The screen is rendered at a framerate of around 59.7 fps. At the start
        of a VBlank period, the PPU framebuffer is pushed to an SDL texture
        and immediately rendered. All of the emulator logic before this is done
        as fast as possible. Since the screen can't be affected during VBlank,
        this is a safe way to render. After all component logic is computed
        at the last cycle of VBlank, the emulator sleeps for time necessary
        to stay at 59.7 fps. This is done by counting the amount of cycles
        that have passed since the last wait period, and the time that it
        occured. If the LCD is turned off by the emulated game, the screen
        renders a blank texture and never updates until the LCD is turned back
        on.

    \section{Closing thoughts}
        I enjoyed working on this project, but it is not yet fully finished.
        The entire APU and serial port are left out. Programming the audio
        processor would probably take a very long time because it is a
        complicated system. The serial port would probably require doing
        some inter-process communication which I have never attempted. Since
        the rest of the project was already complex, I decided to leave these
        features out, since I thought they wouldn't be necessary for games to
        function. The serial port is a bit differnt thought, because one
        specific game randomly refused to work despite my CPU implementation
        passing all test ROM tests, and I think it's because the serial port
        isn't implemented.

        As well as these, no MBCs (memory bank controllers) are implemented.
        In real cartridges, these take care of mapping bigger ROMs to the small
        range offered by the Game Boy. They also take care of any potential RAM.
        Leaving these out makes many games unplayable, and they would be the
        first thing I would implement if I ever came back to this project.
        I didn't do it now because I couldn't think of an elegant way to
        implement them, so now I only have a skeleton that supports no MBC.

        Overall though, I really enjoyed working on this project. My two
        previous project were an emulator of the 6502 processor. This project
        felt like a natural progression from this, since it involved emulating
        a CPU, and integrating it into a full system. The other components were
        harder to crack, since I never did anything like them before, but I
        managed to get it all working in the end.

    \section{Further Reading}\label{sec:furtherReading}
        \href{https://gbdev.io/pandocs/#foreword}{Pandocs}, a great resource for most Game Boy knowledge.\\
        \href{https://gekkio.fi/files/gb-docs/gbctr.pdf}{Gekkio Game Boy docs}, more in-depth information, mostly about the CPU
        \href{https://ia903208.us.archive.org/9/items/GameBoyProgManVer1.1/GameBoyProgManVer1.1.pdf}{Official Nintendo Game Boy Programming Manual}

\end{document}
